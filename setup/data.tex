%%% Основные сведения %%%
\newcommand{\thesisAuthorLastName}{Zholobov}
\newcommand{\thesisAuthorOtherNames}{Yefim Vital'evich}
\newcommand{\thesisAuthorInitials}{Ye.~V.}
\newcommand{\thesisAuthor}             % Презентация, ФИО автора
{%
    \texorpdfstring{% \texorpdfstring takes two arguments and uses the first for (La)TeX and the second for pdf
        \thesisAuthorLastName~\thesisAuthorOtherNames% так будет отображаться на титульном листе или в тексте, где будет использоваться переменная
    }{%
        \thesisAuthorLastName, \thesisAuthorOtherNames% эта запись для свойств pdf-файла. В таком виде, если pdf будет обработан программами для сбора библиографических сведений, будет правильно представлена фамилия.
    }
}
\newcommand{\thesisAuthorShort}        % Презентация, ФИО автора инициалами
{\thesisAuthorInitials~\thesisAuthorLastName}
\newcommand{\thesisTitle}              % Презентация, название
{How to make a presentation in Latex}
\newcommand{\thesisAuthorLevel}    % Презентация, статус (должность), номер
{Bachelor}
\newcommand{\thesisSpecialtyNumber}    % Презентация, направление (специальность), номер
{Number of direction}
\newcommand{\thesisSpecialtyTitle}     % Презентация, направление (специальность), название (Согласно требованиям СПБГУ https://edu.spbu.ru/images/data/normativ_acts/local/20180703_6616_1.pdf)
{Name of direction}
\newcommand{\thesisProgramNumber}    % Презентация, программа, номер
{Number of education program}
\newcommand{\thesisProgramTitle}     % Презентация, программа, название (Согласно требованиям СПБГУ https://edu.spbu.ru/images/data/normativ_acts/local/20180703_6616_1.pdf)
{Name of education program}
\newcommand{\thesisCity}               % Презентация, город написания диссертации
{St. Petersburg}
\newcommand{\thesisYear}               % Презентация, год написания диссертации
{\the\year}
\newcommand{\thesisOrganization}       % Презентация, организация
{St. Petersburg State University}
\newcommand{\thesisOrganizationShort}  % Презентация, краткое название организации для доклада
{SPbU}

\newcommand{\thesisInOrganization}     % Презентация, организация в предложном падеже: Работа выполнена в ...
{Name of Organization}

\newcommand{\supervisorFio}              % Научный руководитель, ФИО
{Смирнов Николай Васильевич}
\newcommand{\supervisorRegaliaPos}          % Научный руководитель, регалии
{профессор}
\newcommand{\supervisorRegaliaDeg}          % Научный руководитель, регалии
{доктор физ.-мат. наук}
\newcommand{\supervisorCathedra}         % Научный руководитель, ФИО
{кафедра \\ моделирования экономических \\ систем}
\newcommand{\supervisorFioShort}         % Научный руководитель, ФИО
{N.\,V.~Smirnov}
\newcommand{\supervisorRegaliaShort}     % Научный руководитель, регалии
{д.~физ.мат.~наук,~проф.}


%\newcommand{\supervisorFio}              % Научный руководитель, ФИО
%{Смирнов Николай Васильевич}
%\newcommand{\supervisorRegaliaPos}          % Научный руководитель, регалии
%{профессор}
%\newcommand{\supervisorRegaliaDeg}          % Научный руководитель, регалии
%{доктор физ.-мат. наук}
%\newcommand{\supervisorCathedra}         % Научный руководитель, ФИО
%{кафедра \\ моделирования экономических \\ систем}
%\newcommand{\supervisorFioShort}         % Научный руководитель, ФИО
%{Н.\,В.~Смирнов}
%\newcommand{\supervisorRegaliaShort}     % Научный руководитель, регалии
%{д.~физ.мат.~наук,~проф.}

%% \newcommand{\supervisorTwoDead}{}        % Рисовать рамку вокруг фамилии
%% \newcommand{\supervisorTwoFio}           % Второй научный руководитель, ФИО
%% {\fixme{Фамилия Имя Отчество}}
%% \newcommand{\supervisorTwoRegalia}       % Второй научный руководитель, регалии
%% {\fixme{уч. степень, уч. звание}}
%% \newcommand{\supervisorTwoFioShort}      % Второй научный руководитель, ФИО
%% {\fixme{И.\,О.~Фамилия}}
%% \newcommand{\supervisorTwoRegaliaShort}  % Второй научный руководитель, регалии
%% {\fixme{уч.~ст.,~уч.~зв.}}

\newcommand{\opponentFio}              % Оппонент 1, ФИО
{Смирнов Николай Васильевич}
\newcommand{\opponentRegaliaPos}          % Оппонент 1, регалии
{профессор}
\newcommand{\opponentRegaliaDeg}          % Оппонент 1, регалии
{доктор физ.-мат. наук}
\newcommand{\opponentCathedra}         % Оппонент 1, ФИО
{кафедра \\ моделирования экономических \\ систем}
%\newcommand{\opponentJobPlace}      % Оппонент 1, место работы
%{Не очень длинное название для места работы}
\newcommand{\opponentFioShort}         % Оппонент 1, ФИО
{Н.\,В.~Смирнов}
\newcommand{\opponentRegaliaShort}     % Оппонент 1, регалии
{д.~физ.мат.~наук,~проф.}

%\newcommand{\opponentOneFio}           % Оппонент 1, ФИО
%{Фамилия Имя Отчество}
%\newcommand{\opponentOneRegalia}       % Оппонент 1, регалии
%{доктор физико-математических наук, профессор}
%\newcommand{\opponentOneJobPlace}      % Оппонент 1, место работы
%{Не очень длинное название для места работы}
%\newcommand{\opponentOneJobPost}       % Оппонент 1, должность
%{старший научный сотрудник}

%\newcommand{\opponentTwoFio}           % Оппонент 2, ФИО
%{\fixme{Фамилия Имя Отчество}}
%\newcommand{\opponentTwoRegalia}       % Оппонент 2, регалии
%{\fixme{кандидат физико-математических наук}}
%\newcommand{\opponentTwoJobPlace}      % Оппонент 2, место работы
%{\fixme{Основное место работы c длинным длинным длинным длинным названием}}
%\newcommand{\opponentTwoJobPost}       % Оппонент 2, должность
%{\fixme{старший научный сотрудник}}

%% \newcommand{\opponentThreeFio}         % Оппонент 3, ФИО
%% {\fixme{Фамилия Имя Отчество}}
%% \newcommand{\opponentThreeRegalia}     % Оппонент 3, регалии
%% {\fixme{кандидат физико-математических наук}}
%% \newcommand{\opponentThreeJobPlace}    % Оппонент 3, место работы
%% {\fixme{Основное место работы c длинным длинным длинным длинным названием}}
%% \newcommand{\opponentThreeJobPost}     % Оппонент 3, должность
%% {\fixme{старший научный сотрудник}}

\providecommand{\keywords}%            % Ключевые слова для метаданных PDF диссертации и автореферата
{}
